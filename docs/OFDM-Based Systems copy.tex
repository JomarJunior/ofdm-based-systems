\documentclass[conference]{IEEEtran}
% Add the compsoc option for Computer Society conferences.
%
% If IEEEtran.cls has not been installed into the LaTeX system files,
% manually specify the path to it like:
% \documentclass[conference]{../sty/IEEEtran}

% Some very useful LaTeX packages include:
% (uncomment the ones you want to load)

% *** MISC UTILITY PACKAGES ***
%\usepackage{ifpdf}
% Heiko Oberdiek's ifpdf.sty is very useful if you need conditional
% compilation based on whether the output is pdf or dvi.
% usage:
% \ifpdf
%   % pdf code
% \else
%   % dvi code
% \fi

% *** CITATION PACKAGES ***
\usepackage{cite}
% cite.sty was written by Donald Arseneau
% V1.6 and later of IEEEtran pre-defines the format of the cite.sty package
% \cite{} output to follow that of the IEEE. Loading the cite package will
% result in citation numbers being automatically sorted and properly
% "compressed/ranged". e.g., [1], [9], [2], [7], [5], [6] without using
% cite.sty will become [1], [2], [5]--[7], [9] using cite.sty.

% *** GRAPHICS RELATED PACKAGES ***
\usepackage{graphicx}
\graphicspath{{../images/}}
% declare the path(s) where your graphic files are
% \graphicspath{{../pdf/}{../jpeg/}}
% and their extensions so you won't have to specify these with
% every instance of \includegraphics
% \DeclareGraphicsExtensions{.pdf,.jpeg,.png}

% *** MATH PACKAGES ***
\usepackage{amsmath}
% A popular package from the American Mathematical Society that provides
% many useful and powerful commands for dealing with mathematics.

% *** SPECIALIZED LIST PACKAGES ***
%\usepackage{algorithmic}
% algorithmic.sty was written by Peter Williams and Rogerio Brito.
% This package provides an algorithmic environment for describing algorithms.
% You can use the algorithmic environment in-text or within a figure
% environment to provide for a floating algorithm.

% *** ALIGNMENT PACKAGES ***
%\usepackage{array}
% Frank Mittelbach's and David Carlisle's array.sty patches and improves
% the standard LaTeX2e array and tabular environments to provide better
% appearance and additional user controls. As the default LaTeX2e table
% generation code is lacking to the point of almost being broken with
% respect to the quality of the end results, all users are strongly
% advised to use an enhanced (at the very least that provided by array.sty)
% set of table tools.

% *** SUBFIGURE PACKAGES ***
%\ifCLASSOPTIONcompsoc
%  \usepackage[caption=false,font=normalsize,labelfont=sf,textfont=sf]{subfig}
%\else
%  \usepackage[caption=false,font=footnotesize]{subfig}
%\fi
% subfig.sty, written by Steven Douglas Cochran, is the modern replacement
% for subfigure.sty, the latter of which is no longer maintained and is
% incompatible with some LaTeX packages including fixltx2e. However,
% subfig.sty requires and automatically loads Axel Sommerfeldt's caption.sty
% which will override IEEEtran.cls' handling of captions and this will result
% in non-IEEE style figure/table captions. To prevent this problem, be sure
% and invoke subfig.sty's "caption=false" package option.

% *** FLOAT PACKAGES ***
%\usepackage{fixltx2e}
% fixltx2e, the successor to the earlier fix2col.sty, was written by
% Frank Mittelbach and David Carlisle. This package corrects a few problems
% in the LaTeX2e kernel, the most notable of which is that in current
% LaTeX2e releases, the ordering of single and double column floats is not
% guaranteed.

%\usepackage{stfloats}
% stfloats.sty was written by Sigitas Tolusis. This package gives LaTeX2e
% the ability to do double column floats at the bottom of the page as well
% as the top.

% *** PDF, URL AND HYPERLINK PACKAGES ***
%\usepackage{url}
% url.sty was written by Donald Arseneau. It provides better support for
% handling and breaking URLs. url.sty is already installed on most LaTeX
% systems. The latest version and documentation can be obtained at:
% http://www.ctan.org/pkg/url
% Basically, \url{my_url_here}.

\usepackage{listings}
\usepackage{xcolor}

\lstset{
  basicstyle=\ttfamily\small,
  breaklines=true,
  breakatwhitespace=true,
  frame=single,
  columns=fullflexible,
  keepspaces=true,
  showstringspaces=false,
  commentstyle=\color{gray},
  keywordstyle=\color{blue},
  stringstyle=\color{orange}
}

% correct bad hyphenation here
\hyphenation{op-tical net-works semi-conduc-tor}


\begin{document}
%
% paper title
% Titles are generally capitalized except for words such as a, an, and, as,
% at, but, by, for, in, nor, of, on, or, the, to and up, which are usually
% not capitalized unless they are the first or last word of the title.
% Linebreaks \\ can be used within to get better formatting as desired.
% Do not put math or special symbols in the title.
\title{Orthogonal Frequency Division Multiplexing (OFDM) Based Systems}


% author names and affiliations
% use a multiple column layout for up to three different
% affiliations
\author{\IEEEauthorblockN{Jomar Júnior de Souza Pereira}
\IEEEauthorblockA{Federal University of Rio de Janeiro\\
Department of Electronic and Computer Engineering\\
Rio de Janeiro, Brazil\\
Email: jomarjunior@poli.ufrj.br}}

% conference papers do not typically use \thanks and this command
% is locked out in conference mode. If really needed, such as for
% the acknowledgment of grants, issue a \IEEEoverridecommandlockouts
% after \documentclass

% for over three affiliations, or if they all won't fit within the width
% of the page, use this alternative format:
% 
%\author{\IEEEauthorblockN{Michael Shell\IEEEauthorrefmark{1},
%Homer Simpson\IEEEauthorrefmark{2},
%James Kirk\IEEEauthorrefmark{3}, 
%Montgomery Scott\IEEEauthorrefmark{3} and
%Eldon Tyrell\IEEEauthorrefmark{4}}
%\IEEEauthorblockA{\IEEEauthorrefmark{1}School of Electrical and Computer Engineering\\
%Georgia Institute of Technology,
%Atlanta, Georgia 30332--0250\\ Email: see http://www.michaelshell.org/contact.html}
%\IEEEauthorblockA{\IEEEauthorrefmark{2}Twentieth Century Fox, Springfield, USA\\
%Email: homer@thesimpsons.com}
%\IEEEauthorblockA{\IEEEauthorrefmark{3}Starfleet Academy, San Francisco, California 96678-2391\\
%Telephone: (800) 555--1212, Fax: (888) 555--1212}
%\IEEEauthorblockA{\IEEEauthorrefmark{4}Tyrell Inc., 123 Replicant Street, Los Angeles, California 90210--4321}}

% make the title area
\maketitle

% As a general rule, do not put math, special symbols or citations
% in the abstract
\begin{abstract}
This report documents recent enhancements to a Python-based Orthogonal Frequency Division Multiplexing (OFDM) simulator used for evaluating cyclic-prefix (CP) and zero-padding (ZP) schemes over frequency-selective channels. We focus on the Lin-Phoong P2 impulse response while sweeping prefix-to-channel ratios, equalization strategies, and signal-to-noise ratio (SNR) points. New automation produces organized bit-error-rate (BER) curves and constellation plots for Zero Forcing (ZF) and Minimum Mean-Square Error (MMSE) equalizers, enabling a quantitative comparison of how relaxed prefix-length constraints (ratio values above one) impact detection performance.
\end{abstract}

% no keywords
\begin{IEEEkeywords}
wireless communications, OFDM, LTE, 5G, 6G, simulation, python
\end{IEEEkeywords}


% For peer review papers, you can put extra information on the cover
% page as needed:
% \ifCLASSOPTIONpeerreview
% \begin{center} \bfseries EDICS Category: 3-BBND \end{center}
% \fi
%
% For peerreview papers, this IEEEtran command inserts a page break and
% creates the second title. It will be ignored for other modes.
\IEEEpeerreviewmaketitle



\section{Introduction}
Orthogonal Frequency Division Multiplexing (OFDM) is the modulation workhorse of many broadband wireless standards because it converts frequency-selective channels into a set of narrowband subcarriers that are easier to equalize. Guard intervals based on cyclic prefixes (CP) or zero padding (ZP) are essential to mitigate inter-symbol interference and to preserve subcarrier orthogonality.

The Python toolkit developed for this project enables rapid exploration of OFDM configurations by combining configurable bit generation, modulation, prefix handling, channel convolution, and linear equalization blocks. The latest iteration extends the allowable prefix-to-channel ratio beyond one, making it possible to assess whether allocating more samples than the nominal channel order improves performance. This document summarizes the resulting simulation campaign and provides references to the generated artifacts so that the results can be replicated or incorporated into future analyses.

\section{System Model}
The simulator generates independent and identically distributed random bits, maps them onto 64-state quadrature amplitude modulation (64-QAM) symbols, and groups the symbols into blocks of 64 subcarriers. After serial-to-parallel conversion, a uniform power allocation is applied before the inverse fast Fourier transform (IFFT). Guard intervals are inserted using either a cyclic prefix (CP) or zero padding (ZP). The transmitted waveform propagates through the custom Lin-Phoong P2 channel, which is represented by a four-tap complex-valued impulse response loaded from \texttt{config/channel\_models/Lin-Phoong\_P2.npy}. Additive white Gaussian noise (AWGN) is injected according to the requested SNR point.

At the receiver, the guard interval is removed, the fast Fourier transform (FFT) is computed, and equalization is carried out with either Zero Forcing (ZF) or Minimum Mean-Square Error (MMSE) operators. Detected symbols are normalized to unit average power before constellation demapping produces the recovered bit stream. Bit-error rate (BER) metrics are computed by comparing against the transmitted bits, and diagnostic plots (constellations and BER versus SNR) are produced for each simulation trial.

\section{Simulation Setup}
Table~\ref{tab:sim_params} lists the common settings used across all sweeps. Each configuration sweeps seven SNR points between 0~dB and 30~dB in 5~dB increments. The parameter \texttt{prefix\_length\_ratio} is multiplied by the channel order and truncated to an integer, so the ratios $\{0.34,\,0.68,\,1.00,\,1.34\}$ yield guard intervals of one through four complex samples. For every run the resulting CSV files are archived under \texttt{results/ber\_*} and the figures are grouped in folders beneath \texttt{images/Lin-Phoong\_P2/} according to the guard interval index.

\begin{table}[!t]
  \caption{Simulation parameters}
  \label{tab:sim_params}
  \centering
  \begin{tabular}{ll}
    \hline
    Parameter & Value \\
    \hline
    Channel model & Lin-Phoong P2 (4 taps) \\
    Subcarriers per OFDM symbol & 64 \\
    Modulation & 64-QAM \\
    Random bits & 6{,}000{,}000 per run \\
    Equalizers & ZF and MMSE \\
    Prefix schemes & CP and ZP \\
    Prefix ratios & 0.34, 0.68, 1.00, 1.34 \\
    SNR sweep & $0{:}5{:}30$ dB \\
    Power allocation & Uniform \\
    Noise model & AWGN \\
    \hline
  \end{tabular}
\end{table}

\section{Results and Discussion}
\subsection{Equalization and Prefix Strategy}
Tables~\ref{tab:ber_zf} and~\ref{tab:ber_mmse} summarize the BER observed at 30~dB for each guard interval length. Under ZF equalization, both CP and ZP variants exhibit nearly identical performance because noise enhancement dominates the residual inter-symbol interference once two or more guard samples are inserted. Switching to MMSE equalization reduces the BER by roughly an order of magnitude, bringing the 30~dB error rate down to approximately $2\times10^{-2}$ when three or four guard samples are used. The difference between CP and ZP remains within $1\times10^{-4}$ across the entire sweep, confirming that the choice between the two guard strategies is largely driven by implementation constraints rather than link-quality considerations in this scenario.

\begin{table}[!t]
  \caption{ZF equalization: BER at 30~dB versus guard interval configuration}
  \label{tab:ber_zf}
  \centering
  \begin{tabular}{lccc}
    \hline
    Prefix & Ratio & Length & BER \\
    \hline
    CP & 0.34 & 1 & 0.3384 \\
    CP & 0.68 & 2 & 0.2423 \\
    CP & 1.00 & 3 & 0.1779 \\
    CP & 1.34 & 4 & 0.1772 \\
    ZP & 0.34 & 1 & 0.3385 \\
    ZP & 0.68 & 2 & 0.2418 \\
    ZP & 1.00 & 3 & 0.1781 \\
    ZP & 1.34 & 4 & 0.1770 \\
    \hline
  \end{tabular}
\end{table}

\begin{table}[!t]
  \caption{MMSE equalization: BER at 30~dB versus guard interval configuration}
  \label{tab:ber_mmse}
  \centering
  \begin{tabular}{lccc}
    \hline
    Prefix & Ratio & Length & BER \\
    \hline
    CP & 0.34 & 1 & 0.0410 \\
    CP & 0.68 & 2 & 0.0266 \\
    CP & 1.00 & 3 & 0.0189 \\
    CP & 1.34 & 4 & 0.0189 \\
    ZP & 0.34 & 1 & 0.0411 \\
    ZP & 0.68 & 2 & 0.0268 \\
    ZP & 1.00 & 3 & 0.0190 \\
    ZP & 1.34 & 4 & 0.0190 \\
    \hline
  \end{tabular}
\end{table}

\subsection{Prefix Length Sensitivity}
Extending the guard interval beyond the channel order provides diminishing returns. For the ZP--MMSE configuration, increasing the prefix ratio from 0.34 to 0.68 halves the 30~dB BER, but further increases toward 1.34 produce only marginal gains. Figure~\ref{fig:zpmmsel} shows the full BER curve for the ratio~1.00 case (three guard samples) and is representative of the behavior observed for the longer guard intervals. The corresponding PNG files reside in \texttt{images/Lin-Phoong\_P2/ZP-OFDM-MMSE-Kn/}, while the CSV snapshots used for this analysis are stored alongside the simulator output under \texttt{results/}.

\begin{figure}[!t]
  \centering
  \includegraphics[width=0.48\textwidth]{Lin-Phoong_P2/ZP-OFDM-MMSE-K3/ZP-OFDM-MMSE-64QAM-UNIFORM-BER_vs_SNR.png}
  \caption{BER versus SNR for ZP-OFDM with MMSE equalization and prefix ratio 1.00 (three guard samples).}
  \label{fig:zpmmsel}
\end{figure}

\section{Conclusion}
The expanded simulation campaign confirms that MMSE equalization is essential for high-SNR operation over the Lin-Phoong P2 channel when 64-QAM is employed. Allowing prefix ratios above one enables up to four guard samples, yet the data shows that three samples already capture most of the achievable performance. Both CP and ZP implementations behave similarly when the guard interval spans at least the channel memory, so the preferred choice can hinge on transmitter or receiver complexity. The curated set of CSV files and figures now stored beneath \texttt{results/} and \texttt{images/} provides a reproducible baseline for future investigations, including potential extensions to adaptive modulation or water-filling strategies.

\bibliographystyle{IEEEtran}
\bibliography{IEEEabrv,references}

\end{document}